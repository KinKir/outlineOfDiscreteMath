
\chapter* {Introduction}
Discrete math is a survey of many different areas of mathematics not covered by the usual series of calculus classes required of an engineering student. Since it is a survey it uses an overwhelming number of conventional symbols that have been adopted over the centuries by mathematicians. You will be reading a paper that was set using a popular free typesetting tool called Latex that is nearly universally used by scientists, engineers and mathematicians to publish papers. At times in this paper you will see references to backslashed sequences like $\forall$, $\in$, etc. These are references to the plaintext markups that are used in Latex in case you want to begin using that tool. For anyone who aspires to graduate school it is a requirement.


While the intent of this material is to be as purely formal as possible, it is necessary to make the course easier to make some assumptions of prior knowledge even before it is formally introduced. For example the section on logic assumes the reader is already familiar with all secondary school math including the concepts of integer versus real numbers and the basic rules of algebra.

In order to make a course for this material as compact and efficient as possible, this outline makes many assumptions about prerequisite material. It is assumed the student is already familiar with a two column proof form as typically presented in a high school geometry class. At least two courses in programming are assumed with a thorough understanding of the logic of Boolean expressions and their evaluation. The most basic understanding of function and set are taken as a given to avoid introducing more formal material on those topics. Directed graphs are introduced when talking about relations even though the formal treatment is later. And any material which requires calculus is excluded even though the student is expected to have the algebra skills that are needed to complete at least one course in calculus requires.

For an intense 6 week summer course there is always a chance that all of the material in this outline will not be covered. In our opinion the section on discrete probability can be safely skipped when the program will require a full course in statistics and probability. And the section on trees can be skipped since there will be a programming course on data structures and the student who completes this section on graphs should be well prepared to quickly master the new material in that course. An instructor using this material can of course take the chapters in a different order although there was great attention given this outline to avoid introducing material out of this sequence. 

\section*{Comment About Copyright}
This text must be treated as a derivative work of Rosen unless and until a publisher picks up this manuscript. I believe that it can be used under a theory of fair use for classroom use but that is not a legal opinion, only a self-serving moralistic one. I would encourage anyone who wishes to use this consider the policies of their institution before making copies readily available. 

\section*{Comments About the Order of the Material}
There are two ways to approach this material in the courses I have seen, start with logic or start with sets. I believe that starting with logic makes more sense for two reasons. First, a computer science student will have already achieved a level of mastery over Boolean expressions from programming and will be comfortable with the introduction of propositional calculus. Second, for those who wish to take a more formal approach to the material, they can get through the section on logic and may be able to largely skip the section on proofs and achieve that end. To make this text more approachable for those at a teaching college we have chosen to cover logic first to make the discussion of proofs easier.

Many instructors like to introduce relations before functions and in many academic ways it makes more sense. However every student has been well introduced to functions before they take this course but only introduced to relations in a less rigorous way. By giving them material that they think they know but developing a more formal approach to it, the fact that relations can be viewed as a superset of functions should make it easier for them to grasp the formalism. 

Many instructors will place the material on Graphs and Trees at the end of the course. We believe this is a mistake. Material covered at the end of the course is often neglected and the applications of graphs and trees are essential in industry and in later computer science courses. We bring it forward to ensure that there is sufficient time to study them in depth. This is done to the detriment of discrete probability. We believe that every computer science program will require a course in probability and statistics and find that putting this material at the end does not impact the flow of presentation while recognizing the value of giving a student some exposure to the discrete form of probability before that undergraduate material. 

We use a similar reason to push the material on computation to the end. Many courses do not even take up this subject in favor of covering it as a complete course by itself. We include it here to offer some flexibility to the instructor who may choose to cover some material without trying to cover all the material we provide. 

We intend for this text to be a competitor to Schaum's Outline which we have judged to be insufficient for our course offerings. This text has far greater depth than Schaums's requiring less supplementation for the instructor. 

\section* {Stylistic and typographic quirks}
Style manuals specify that a period at the end of a sentence be included within the parenthetic expression when it ends the sentence. We reject that and go with the non-standard usage of placing the period after the closing parethesis. Likewise for quotes, commas, etc. The attempt is to bring the English into alignment with programming practice instead of obedience to an illogical (from a CS perspective) convention.

\section*{Some History}
Previous to the 1970s, there were no computer science programs to speak of. Those who were the computer science pioneers were mathematicians first and practiced computer science as applied mathematics. Consequently they were well educated in many areas of mathematics that are truncated or absent in many computer science programs as the major tends to cater more toward the needs of software engineers instead of applied mathematicians. 

At its best, a course in discrete mathematics for the computer scientist must cover a great deal of material in a very short period of time. This is compounded by its placement in most programs in the first two years of undergraduate study. The need to teach it quickly and thoroughly is a challenge. This particular text was originally created to serve as a reference text to replace dependence upon Schaum's or one of the many texts offered. In our opinion those texts suffered from one of two defects, they were either encyclopedic and expensive or they left out key material that required supplementation by the instructor. This text at least answers the second problem and arguably the first in some applications. By reducing the exercises and most of the proofs for the theorems it is far shorter than the scope of material would allow if it attempted to take on the burden of being a teaching text. So it still requires a great deal of supplement for exercises and lectures. However for a student reference work to use during and after the course we believe it is superior to any alternative. And of course it is being offered for free to fellow instructors of this material. The assumption is that some instructors, and perhaps even some students, will offer constructive criticism on how the mission of this text can be improved for future instructors and students.

It may seem a bit pompous to call this a forward to the first edition. But that is simply done to acknowledge that as a first edition this text is surely imperfect in many ways. Most obviously there are inevitable typographical errors from transcription or flaws in the language. But it is also a hope that we will return to this and offer other volumes connected to this which include the much needed exercises and even some courseware to make the job of an instructor charged with teaching this material slightly easier.
