



                %%%   PROBABILITY   %%%
\chapter {Discrete Probability}
\section {An Introduction to Discrete Probability}
    \subsection {Finite Probability}
    \subsection {The Probability of Combinations of Events}
    \subsection {Probabilistic Reasoning}

\section {Probability Theory}
    \subsection {Assigning Probabilities}
    \subsection {Combinations of Events}
    \subsection {Conditional Probability}
    \subsection {Independence}
    \subsection {Bernoulli Trials and the Binomial Distrubution}
    \subsection {The Birthday Problem}
    \subsection {Monte Carlo Algorithms}
    \subsection {The Probabilistic Method}






Let p be the proposition that the sum of the first n odd numbers is n**2. How can we prove such a proposition? Here is the example of how inductive reasoning works. 

We can easily evaluate this proposition for small values of n and see that they are true. But since the set of input values is the set of natural numbers we cannot do this for all elements of the set. So we observe this, let us assume that this proposition is true for some value k which is bigger than any value we evaluated manually. If we can prove that the statement MUST be true for the next value, the successor of k, k+1, then we have proven that it must be true for ALL values of n drawn from the natural numbers since we know it is true for the small values and we can continually apply the reasoning that got us from k to k+1 as many times as we need to give us all the values to infinity.

So first we introduce the inductive hypothesis, that it is true for some k:
Assume that the first k odd numbers sum to k**2. Now we have the proof obligation to prove that with that assumption that this MUST be true for k+1, that is, the sum of the first k+1 odd numbers will give us (k+1)**2. This requires some clever algebra but nothing you can't follow:

the first k odd number sum to k**2
Sigma(i=0 to k, 2i+1) = Sigma(i=1 to k-1, 2i+1) = k**2
which is equivalent to 
1+3+5+ ... +2(k-1) = k**2
we add 2(k+1) to both sides of the equation giving
1+3+5+ ... +2(k-1)+2(k+1) = (k+1)**2 = (k**2 + 2k + 1)=k**2 + (2k+1)
Using the premise, we rewrite the LHS
1+3+5+ ... +2(k-1)  + 2(k+1) = k**2 + 2(k+1)= k**2 + 2k + 1
showing the left and RHS of the equation are equal QED.

The general principle of weak form of mathematical induction is
First, show the proposition is true for one small element from the input.
Show that IF the proposition is true for some arbitrary k, that it MUST also be true for the next value after k, k+1. 
After both parts are proven, you have proven for all value from N.

CAUTION: The inductive assumption looks similar to the thing to be proven but you may not use that in the argument. You must use an arbitrary value k, and then prove that it must also be true for k+1 without again stating the assumption. To do so is the famous logical fallacy of assuming the antecedent or begging the question. This is a common error in inductive proofs.

There are a set of proofs that can be solved inductively but require that more than one small value be proven. This leads to the stronger form of inductive reasoning. In the strong form, you must prove that the proposition holds for some small number of values.

Open Form versus Closed Form Solutions
Note that we have proved an equivalence between two expressions, the sum of the first n odd numbers and the expression n**2. The first form has the implied algorithm of summing the first n odd integers, something that is of O(n) while the second has O(1). We call the first version an open form solution while we call the second a closed form solution. The computational advantage is obvious.

\newpage