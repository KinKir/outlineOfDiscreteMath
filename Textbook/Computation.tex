

           %%% COMPUTATION and FORMAL LANGUAGES  %%%
                                                                 
% This chapter can be skipped for programs that require a Theory of Computation Course that does not have a prerequisite
(Excluded from UCD course offerings)
\chapter {Computation}
Association between Automata, Grammar, Language. Difference between syntax and semantics in natural language. Object first or object last in natural language. 

\section {Languages and Grammars}
We saw that the notation $A_*$ designates all the possible strings that can be constructed from the set $A$. When the set A contains symbols that are distinguishable from each other we call that set an $\textit{alphabet}$ and refer to it with the greek letter $\Sigma$. Since there is no ambiguity between the strings $(a_1,a_2, \dots a_n)$ and $a_1a_2a_3 \dots a_n$ for strings of length $n$, we adopt this notation for string sequences. For short strings then
$$(w,o,r,d) = word$$
We call such strings words, the symbols from the alphabet lettersand the finite sequences in $\Sigma_*$ as the strings or words generated by the letters of $\Sigma$.

Alphabet, Words
Operations on words: concatenation
Formal Definition of Language
Operations on Languages
    \subsection {Phrase-Structured Grammars}
Regular Expressions, Regular Languages
Generative Grammars, Rules of Production
    \subsection {Context Free and Context Sensitive Grammars}
    \subsection {Regular Languages and their Notation}
    \subsection {Derivation Trees}
    parsing
    \subsection {Backus-Naur Form}
    
\section {Finite State Machines}
States, State Transition, Finite State Automata
    \subsection {FSM with no output}
        \subsubsection {Set of Strings}
        \subsubsection {Finite-State Automata}
        Def 3: A \textit{finite-state automata} $M=(S,I,f,s_0,F)$ consists of a finite set $S$ of \textit{states}, a finite \textit{input alphabet} $I$, a \textit{transition function} $f$ that assigns a next state to every pair of state and input (so that $f:S \times I \rightarrow S$, an \textit{initial} or \textit{start state} $s_0$, and a subset $F$ of $S$ consisting of \textit{final} (or \textit{accepting states}.
        
        \subsubsection {Language Recognition by FSA}
        \subsubsection {Non-deterministic FSA}
        
    \subsection {FSM with output}
    Mealy and Moore machines
    
\section {Language Recognition}
    \subsection {Regular Sets}
    \subsection {Kleene's Theorem}
    \subsection {Regular Sets and Regular Grammars}
    \subsection {Beyond Regular Languages and FSMs}
A later class will explore grammars beyond regular grammars, the languages they generate and the automata that recognize them.


how do i do a citation? %\cite{Rosen:2002:DMA:579402}


\newpage