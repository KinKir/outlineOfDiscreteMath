


                                                                            %%%%%%%%%%%    TREES    %%%
\chapter {Trees}
Recursive def of trees, some famous algorithms for trees

\begin{definition}[Tree]
A \textit{tree} is a connected undirected graph with no simple circuits.
\end{definition}

\begin{theorem}
An undirected graph is a tree if and only if there is a unique simple path between any two of its vertices.
\end{theorem}

\begin{definition}[Rooted Tree]
A \textit{rooted tree} is a tree in hwich one vertex has been designated as the root and every edge is directed away from the root. Suppose $T$ is a rooted tree. If $v$ is a vertex in $T$ other than the root, the \textbf{parent} of $v$ is the unique vertex $u$ such that there is a directed edge from $u$ to $v$. When $u$ is the parent of $v$, $v$ is called a \textbf{child} of $u$. Vertices with the same parent are called \textbf{siblings}. The \textbf{ancestors} of a vertex other than the root are the vertices in the path from the root to this vertex, excluding the vertex itself but including the root. The \textbf{descendants} of a vertex $v$ are those vetices that have $v$ as an ancestor. A vertex of a tree is called a \textbf{leaf} if it has no children. Vertices that have children are called \textbf{internal vertices}. The root is an internal vertex unless it is the only vertex in the graph, in which case it is a leaf. If $a$ is a vertexin a tree, the \textbf{subtree} with $a$ as its root is the subgraph of the tree consisting of $a$ and its descendants and all edges incident to those descendants.
\end{definition}

\begin{definition}
A rooted tree is called an \textit{m-ary tree} if every internal vertex has no more than $m$ children. The tree is called a "\textit{full m-ary tree}" if every internal vertex has exactly $m$ children. An \textit{m}-ary tree with $m=2$ is called an \textit{binary tree}. 
\end{definition} 

\begin{definition}
An \textbf{ordered rooted tree} is a rooted tree where the children of each internal vertex are ordered. Orered rooted trees are drawn so that the hcildren of each internal vertex are shown in order from left to right. Note that a representation of a rooted tree in the conventional way determines an ordering for its edges. 

In an orderedf binary tree (usually called a \textbf{binary tree}), if an internal vertex has two children, the first child is called the \textbf{left child} and the second child is called the \textbf{right child}. The tree rooted at the left child of a vertex is called the \textbf{left subtree} of this vertex, and the tree rooted at the right child of a vertex is called the \textbf{right subtree} of the vertex. 
\end{definition}

\section{Properties of Trees}

\begin{theorem}
A tree with $n$ vertices has $n-1$ edges.
\end{theorem}

\begin{theorem}
A full $m$-ary tree with $i$ internal vertices contains $n=mi+1$ vertices.
\end{theorem}

\begin{theorem}
A full $m$-ary tree with 
\begin{enumerate}[label=(\roman*)]
  \item $n$ vertices has $i=(n-1)/m$ internal vertices and $l=[(m-1)n+1]/m$ leaves,
  \item $i$ internal vertices has $n=mi+1$ vertices and $l=(m-1)i+1$ leaves,
  \item $l$ leaves have $n=(ml-1)/(m-1)$ vertices and $i=(l-1)(m-1)$ internal vertices.
\end{enumerate}
\end{theorem}

\begin{definition}
The level of a vertex $v$ in a rooted tree is the length of the unique path from the root to this vertex. The level of the root is defined to be zero. The \textbf{height} of a rooted tree is the maximum of the levels of vertices. In other words, the height of a rooted tree is the length of the longest path from the root to any vertex. A rooted $m$-ary tree of hgitht $h$ is \textbf{balanced} if all leaves are at levels $h$ or $h-1$.
\end{definition}

\begin{theorem}
There are at most $m^h$ leaves in an $m$-ary tree of height $h$.
\end{theorem}

\begin{corollary}
If an $m$-ary tree of height $h$ has $l$ leaves, then $h \ge \lceil \log_ml \rceil$. If the $m$-ary tree is full and balanced, then $h= \lceil \log_ml \rceil$ 
\end{corollary}

\newpage


