







                                                                              %%%  COUNTING  %%%%%%%%%%%%%%%
\chapter {Counting}
We need to count the operations of an algorithm to comput its time complexity.

\section {The Basics of Counting}
    \subsection {Basic Counting Principles}
product rule, sum rule.
To count means to put the elements in a set into a one-to-one correspondence, a mapping, from the set to the set of natural numbers. That means we find a function $f:\mathbb{N} \rightarrow S$. 
Any set that has a bijection to the natural numbers is said to be \textit{countable}. 


\begin{theorem}[Sum Rule]
 for two disjoint sets A and B
$card(A \cup B) = card(A) + card(B)$
\end {theorem}


\begin{theorem}[Product Rule]
For two disjoint sets A and B
$card(A \times B) = card(A) * card(B)$
\end{theorem}

    \subsection {More Complex Counting Problems}
    \subsection {The Inclusion-Exclusion Principle}
    \subsection {Tree Diagrams}

\section {The Pigeonhole Pinciple}
\begin{theorem}[Pigeonhole Principle]
If $k$ is a positive integer and $k+1$ or more objects are placed into $k$ boxes, then there is at least one box containing two or more of the objects.
\end{theorem}

\begin{corollary}
A function $f$ from a set with $k+1$ elements to a set with $k$ elements is not one-to-one.
\end{corollary}


    \subsection {The Generalized Pigeonhold Principle}
\begin{theorem}[The Generalized Pigeonhole Principle]
If $N$ objects are placed into $k$ boxes, then there is a least one box containing at least $\lceil N/k \rceil$ objects.
\end{theorem}

    \subsection {Some Elegant Applications of the Pigeonhold Principle}
\begin{theorem}
Every sequence of $n^2 + 1$ distinct real numbers contains a subsequence of length $n+1$ that is either strictly increasing or trictly decreasing.
\end{theorem}

\section {Permutations and Combinations}
   
    \subsection {Permutations}
\begin{definition}
A \textbf{permutation} of a set of distinct objects is an ordered arrangement of these objects. An ordered arragement of $r$ elements of a set is called an \textbf{r-permutation}. An $r$-permutation of $r$ objects drawn from a set of $n$ things is denoted as $P(n,r)$.
\end{definition}

\begin{theorem}
If $n$ is a positive integer and $r$ is an integer with $1 \le r \le n$, then there are 
$$P(n,r) = n(n-1)(n-2) \dots (n-r+1)$$
$r$-permutations of a set with $n$ distinct elements.
\end{theorem}

\begin{corollary}
If $n$ and $r$ are integers with $0 \le r \le n$ then $P(n,r)=\frac{n!}{(n-r)!}$.
\end{corollary}


    \subsection {Combinations}
\begin{definition}
A \textbf{combination} of a set of distinct objects is an unorderd arrangement of these objects. An unordered arrangement of $r$ elements of the set is called an \textbf{r-combination}. The number of $r$-combinations of a set with $n$ distinct elements is denoted by $C(n,r)$ also denoted as $n\choose{r}$ and is called the \textbf{binomial coefficient}.
\end{definition}

\begin{theorem}
The number of $r$-combinations of a set with $n$ elements, where $n$ is a nonnegative integer and $r$ is an integer with $0 \le r \le n$, equals
$$C(n,r)=\frac{n!}{r!(n-r)!}$$
\end{theorem}

\begin{corollary}
Let $n$ and $r$ be nonnegative integers with $r \le n$. Then $C(n,r)=C(n,n-r)$.
\end{corollary}

\begin{definition}
A \textit{combinatorial proof} of an identity is a proof that uses counting arguments to prove that both sides of the identtity count the same objects but in different ways.
\end{definition}

\section {Binomial Coefficients}
    \subsection {The Binomial Theorem}
The coefficients of the expansion of a binomial like $(a+b)^n$ are the binomial coefficients. They come up in many different contexts and are the subject of this section.

\begin{theorem}[The Binomial Theorem]
Let $x$ and $y$ be variables, and let $n$ be a nonnegative integer. Then 
$${(x+y)^n} ={ \sum_{j=0}^n}   {{ n} \choose{j}} x^{n-j}y^j$$
$$= {n \choose {0}}  {x^n}   +  
{n \choose 1      }   {x^{n-1}y^1}   +   
{n \choose 2      }   {x^{n-2}y^2} + 
\dots +
{n \choose {n-1}}   {xy^{n-1}} + 
{n \choose n      }    { y^n}$$
\end{theorem}


\begin{corollary}
Let $n$ be a nonnegative integer. then
$$\sum_{k=0}^n {n\choose k} = 2^n$$.
\end{corollary}

\begin{corollary}
Let $n$ be a positive integer. Then
$$\sum_{k=0}^n (-1)^k {n \choose k} = 0$$.
\end{corollary}

\begin{corollary}
Let $n$ be a nonnegative integer. Then
$$ \sum_{k=0}^{n}2^k {n\choose k} = 3^n$$
\end{corollary}

\subsection{Pascal's Identity and Triangle}
\begin{theorem}[Pascal's Identity]
Let $n$ and $k$ be positive integers with $n \ge k$. Then
$$ {{n+1} \choose k}  = {n \choose {k-1}} + { n \choose k}$$.
\end{theorem}



    \subsection {Some Other Identities of the Binomial Coefficients}
\begin{theorem}[Vandermonde's Identity]
Let $m$,$n$, and $r$ be nonnegative integers with $r$ not exceeding either $m$ or $n$. Then 
$$ {{m+n} \choose r} = \sum_{k=0}^r {m\choose{r-k}} {n\choose k}$$.
\end{theorem}

\begin{corollary}
If $n$ is a nonnegative integer, then
$${ {2n}\choose n} = \sum_{k=0}^{n}  {n\choose k}^2$$
\end{corollary}

\begin{theorem}
Let $n$ and $r$ be nonnegative integers with $r \le n$. Then
$${ {n+1} \choose {r+1}} = \sum_{j=r}^n {j\choose r}$$.
\end{theorem}


\section {Generalize Permutations and Combinations}
Until now we selected items exactly once from a set. We now relax that and look at other counting problems.
    \subsection {Permutations with Repetition}
Consider a cash drawer. There are various denominations of bills that are drawn from a small set of bills, typically singles, fives, tens and twenties. It makes no difference what order the bills are in in the cash box since all bills of the same denomination are interchangable (fungible). We put dividers into the cash box to separate the denominations and the minumum number of dividers will be one less than the number of bills. Since we have 4 different denominations, three dividers suffices to separate them. Now consider an alternative representation of the cash box. Since we keep the bills in a set order going smallest to largest and left to right, we know the denomination by its position relative to the dividers. All the singles are before the first divider, the fives between the first and second dividers, etc. So we can represent a bill by a zero and a divider by one and from this representation show any possible cash box.  So this string of zeros and ones represents 5 singles, four fives, three tens and no twenties $000001000010001$. Now recognize that if we want to know how many combinations of $r$ objects we can create from a set of $n$ elements allowing for repetition, that is the same as asking how many combinations of $r$ zeros we can create where we have $n-1$ ones, a problem we previously solved.


\begin{theorem}
The number of $r$-permutations of a set of $n$ objects with repetition allowed is $n^r$.
\end{theorem}


    \subsection {Combinations with Repetition}
\begin{theorem} 
There are $C(n+r-1,r) = c(n+r-1,n-1)$  $r$-combinations from a set for $n$ elements when repetition of elements is allowed.
\end{theorem} 



    \subsection {Permutations with Indistinguishable Objects}
Up to now we have only drawn the objects from a set. A set does not have duplicates although we allowed multiple draws of the same element. Now we consider bags where some elements are indistinguishable from each other much like ping-pong balls in a ball pit. Consider, how many different words can possibly be constructed from the letters of the word 'SUCCESS'? Which of the S's chosen from the bag makes no difference in the constructed word since they are indistinguishable.
\begin{theorem}\label{PermDistinguishedBoxesIndistinguishableObjects}[Permutations with Indistinguishable Objects]
The number of different permutation of $n$ objects, where there are $n_1$ indistinguishable objects of type 1, $n_2$ indistinguishable objects of type 2,  $\dots$, and $n_k$ indistinguishable objects of type $k$, is 
$$\frac{n!}{n_1! n_2! n_3! \dots n_k!}$$
\end{theorem}

    \subsection {Counting Combinations when Distributing Objects into Boxes}
We now consider the general cases of putting objects into boxes. The objects to be distributed may be distinguishable and the boxes into which they are distributed are distinguishable, say the dealt cards in a card game. Sometimes the objects being distributed are indistinguishable, as the bill in a cash drawer, but the boxes are distinguishable, as the slots in the cash drawer are. Sometimes the objects to be distributed are distinguishable, such as the elements of a set, but the boxes are indistinguishable, cubicles in a large office for example. 

\subsubsection {Combinations of Distinguishable objects into distinguishable boxes}
\begin{theorem}
The number of ways to distribute $n$ distinguishable objects into $k$ distinguishable boxes to that $n_i$ objects are placed into box $i$, $i=1,2, \dot ,k$, equals
$$ \frac{n!}{n_1!n_2! \dots n_k!}$$
\end {theorem}

\subsubsection{Combinations of Indistinguishable Objects and Distinguishable Boxes}
\begin{theorem}
There are 
$$C(n+r-1,n-1)$$
ways to place $r$ indistinguishable objects into $n$ distinguishable boxes.
\end{theorem}

\subsubsection{Combinations of Distinguishable Objects and Indistinguishable Boxes}
For example, many office plans have indistinguishable cubicles that may hold multiple people. Let us say there are four new employees to be placed into three cubicles. How many distinct combinations of people in cubicles can there be? You may be surprised to learn there is no simple closed form solution to problems of this kind even though simple cases can easily be worked out by hand. In this example you can say that there is exactly one way to place all for employees into one cube. Then there are four ways to place three employees in one cube with one in another. There are three ways to place two employees in one cube with two in another. In the last case there are six ways to place two employees in one cubicle with the other two in a cubicle by themselves. In total there are 14 combinations. 

\subsubsection{Combinations of Indistinguishable Objects and Indistinguishable Boxes}
An example of this kind of problem is the task of placing 6 books to be shipped into 4 shipping boxes, assuming the books are all of the same title. The books are indistinguishable and the boxes are also indistinguishable. You can place all six in one box, 5 in one and one in another, 4 and 2, 4 and 1 and 1, 3 and 3, 3 and 2 and 1, 3 and 1 and 1, 2 and 2 and 2, 2 and 2 and 1 and 1. 

Observe that distributing $n$ indistinguishable objects into $k$ indistinguishable boxes is the same as writing $n$ as the sum of at most $k$ positive integers in nonincreasing order. If $a_1+a_2+ \dots +a_j=n$, where $a_1,a_2, \dots +a_j$ is a partition of the positive inteers $n$ into $j$ positive integers. We can define a function $p_k(n)$ that gives the number of partitions of $n$ into at most $k$ positive integers. But there is no simple closed form formula for this number.

\section {Generating Permutations and Combinations}
    \subsection {Generating Permutations}
    \subsection {Generating Combinations}





\section{other material, schaum's??}
The size of a set A is called its \textit{cardinality} and is designated $|A|$. Cardinality is defined as the number of elements in the set for a set with a finite number of elements. We say the cardinality of set A is 3 or simply that set A has three elements. If the number of elements in the set $S$ is finite, we call it a finite set. If not we call it an infinite set.

Set Cardinality, Cardinal Numbers, Cardinality of Infinite Sets
The cardinality of a set is defined as the number of elements it contains. Note this implicitly uses the concept of counting which will be more explicitly defined later in the course. When the number of elements in a set is a natural number m or is zero, we say the set is finite in size. Otherwise we say the set has an infinite cardinality or is an infinite set. We define a finite set to be countable. If we can specify a way in which an infinite set can have its members arranged, we call this an infinitely countable set. Later we will see that some sets, such as all the real numbers $\ge 0$ and $\le 1$ are uncountably infinite. 


Theorem: for two disjoint sets A and B not necessarily disjoint
$card(A \cup B)$, the cardinality will be the sum of the cardinality minus the ones double counted because they are in both sets. The ones in both sets are defined as the intersectiion so we have what is called the inclusion, exclusion principle
$card(A \cup B) = card(A) + card(B) - card(A \cap B)$

This can be extended to any number of sets.

The cardinality of infinite sets
The natural numbers are an infinite set. There are many other infinite sets that have a bijection to the set of natural numbers. For example we can use this function to map from natural numbers to the set of integers.  Any set with a bijection to the set of natural numbers is called countably infinite.

Cantor famously proved that the set of rational numbers is countably infinite using the diagonalization argument. We assign a special symbol to the cardinality of all countably infinite sets, aleph null.

Real numbers can be shown to have no bijection possible to natural numbers. We assign a special symbol to the cardinality of the set of real numbers and any other set with a bijection to it as aleph 1.  

There are many sets that are both infinite yet countable. The set of natural numbers are infinite. Since it has a bijection with the natural numbers we call the set of natural numbers \textit{countably infinite}. Any set that has a bijection with the natural numbers is also countably infinite. 

\begin {theorem}
The set $\mathbb{Q}$ is countably infinite. 
\end {theorem}
This is proven using Cantor's diagnolization proof.


    
\newpage