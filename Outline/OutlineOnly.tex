\documentclass[11pt]{book}
\usepackage[utf8]{inputenc}
\usepackage{geometry}
\geometry{letterpaper}

%%% TITLE FORMATS 
\usepackage{sectsty}
%\titleformat{\subsection}
%{\normalfont\slshape}{\thesection}{1em}{}


%%% HEADERS & FOOTERS
\usepackage{fancyhdr} % This should be set AFTER setting up the page geometry
\pagestyle{fancy} % options: empty , plain , fancy
\fancyhf{}
\fancyhead[RO]{An Outline of Discrete Mathematics}
\fancyhead[LE]{\rightmark}
\fancyfoot[LE]{\thepage}
\fancyfoot[RO]{\thepage}
\renewcommand{\headrulewidth}{0pt} % customise the layout...





\begin{document}
\thispagestyle{empty}
\title{A Brief Outline of Discrete Mathematics for the Undergraduate Computer Science Student}
\author{Dale Fletter}
\date{\today} % Activate to display a given date or no date (if empty),
         % otherwise the current date is printed 
         %\thanks{Daryl Posnett, Vladimir Filkov}
\maketitle
\frontmatter 
\tableofcontents
\mainmatter



\chapter{The Foundations: Logic and Proofs}
 \section{Propositional Logic}
  \subsection{Introduction}
  \subsection{Propositions}
  \subsection{Conditional Statements}
  \subsection{Truth Tables of Compound Propositions}
  \subsection{Precedence of Logical Operations}
  \subsection{Logic and Bit Operations}
  
 \section{Applications of Propositional Logic}
  \subsection{Introduction}
  \subsection{Translating English Sentences}
  \subsection{System Specifications}
  \subsection{Boolean Searches}
  \subsection{Logic Puzzles}
  \subsection{Logic Circuits}
  
 \section{Propositional Equivalences}
  \subsection{Introduction}
  \subsection{Logical Equivalneces}
  \subsection{Using De Morgan's Laws}
  \subsection{Constructing New Logical Equivalences}
  \subsection{Propositional Satisfiability}
  \subsection{Applications of Satisfiability}
  \subsection{Solving Satisfiability Problems}
  
 \section{Predicates and Quantifiers}
  \subsection{Introduction}
  \subsection{Predicates}
  \subsection{Quantifiers}
  \subsection{Quantifiers with Restricted Domains}
  \subsection{Precedence of Quantifiers}
  \subsection{Binding Variables}
  \subsection{Logical Equivalences Involving Quantifiers}
  \subsection{Negating Quantified Expressions}
  \subsection{Translating from English into Logical Expressions}
  \subsection{Using Quantifiers in System Specifications}
  \subsection{Examples from Lewis Carroll}
  \subsection{Logic Programming}
  
 \section{Nested Quantifiers}
  \subsection{Introduction}
  \subsection{Understanding Statements Involving Nested Quantifiers}
  \subsection{The Order of Quantifiers}
  \subsection{Translating Mathematical Statements into Statements Involving Nested Quantifiers}
  \subsection{Translating from Nested Quantifiers into English}
  \subsection{Translating English Sentences into Logical Expressions}
  \subsection{Negating Nested Quantifiers}
  
 \section{Rules of Inference}
  \subsection{Introduction}
  \subsection{Valid Arguments in Propositional Logic}
  \subsection{Rules of Inference for Propositional Logic}
  \subsection{Using Rules of Inference to Build Arguments}
  \subsection{Resolution}
  \subsection{Fallacies}
  \subsection{Rules of Inference for Qualtified Statements}
  \subsection{Combining Rules of Inference for Propositions and Quantified Statements}
  
 \section{Introduction to Proofs}
  \subsection{Introduction}
  \subsection{Some Terminology}
  \subsection{Understanding How Theorems Are Stated}
  \subsection{Methods of Proving Theorems}
  \subsection{Direct Proofs}
  \subsection{Proof by Contraposition}
  \subsection{Proofs by Contradiction}
  \subsection{Mistakes in Proofs}
  \subsection{Just a Beginning}
  
 \section{Proof Methods and Strategy}
  \subsection{Introduction}
  \subsection{Exhaustive Proof and Proof by Cases}
  \subsection{Existence Proofs}
  \subsection{Uniqueness Proofs}
  \subsection{Proof Strategies}
  \subsection{Looking for Counterexamples}
  \subsection{Proof Strategy in Action}
  \subsection{Tilings}
  \subsection{The Role of Open Problems}
  \subsection{Additional Proof Methods}
  

\chapter{Basic Structures: Sets, Functions, Sequences, Sums, and Matrices}
 \section{Sets}
  \subsection{Introduction}
   \subsubsection{Definition of Set}
   \subsubsection{Notation for Common Sets}
   \subsubsection{Definition of Set Equality}
   \subsubsection{The Empty Set}
  \subsection{Venn Diagrams}
  \subsection{Subsets}
   \subsubsection{Definition of Subset}
  \subsection{The Size of a Set}
   \subsubsection{Definition of Set Cardinality}
  \subsection{Power Sets}
  \subsection{Cartesian Products}
  \subsection{Using Set Notation with Quantifiers}
  \subsection{Truth Sets and Quantifiers}
  
 \section{Set Operations}
  \subsection{Introduction}
  \subsection{Set Identities}
  \subsection{Generalized Unions and Intersections}
  \subsection{Computer Representation of Sets}
 
 \section{Functions}
  \subsection{Introduction}
  \subsection{One-to-One and Onto Functions}
  \subsection{Inverse Functions and Compositions of Functions}
  \subsection{The Graphs of Functions}
  \subsection{Partial Functions}
  
 \section{Sequences and Summations}
  \subsection{Introduction}
  \subsection{Sequences}
  \subsection{Recurrence Relations}
  \subsection{Special Integer Sequences}
  \subsection{Summations}
  
 \section{Cardinality of Sets}
  \subsection{Introduction}
  \subsection{Countable Sets}
  \subsection{An Uncountable Set}
  
 \section{Matrices}
  \subsection{Introduction}
  \subsection{Matrix Arithmetic}
  \subsection{Transposes and Powers of Matrices}
  \subsection{Zero-One Matrices}
  
 
\chapter{Algorithms}
 \section{Algorithms}
  \subsection{Introduction}
  \subsection{Searching Algorithms}
  \subsection{Sorting}
  \subsection{Greedy Algorithms}
  \subsection{The Halting Problem}
  
 \section{The Growth of Functions}
  \subsection{Introduction}
  \subsection{Big-O Notation}
  \subsection{Big-O Estimates for Some Important Functions}
  \subsection{The Growth of Cobinations of Functions}
  \subsection{Big-Omega and Big-Theta Notation}
  
 \section{Complexity of Algorithms}
  \subsection{Introduction}
  \subsection{Time Complexity}
  \subsection{Complexity of Matrix Multiplication}
  \subsection{Algorithmic Paradigms}
  \subsection{Understanding the Complexity of Algorithms}
 
\chapter{Number Theory and Cryptography}
 \section{Divisibility and Modular Arithmetic}
  \subsection{Introduction}
  \subsection{Division}
  \subsection{The Division Algorithm}
  \subsection{Modular Arithmetic}
  \subsection{Arithmetic Modulo $m$}
  
 \section{Integer Representation and Algorithms}
  \subsection{Introduction}
  \subsection{Representations of Integers}
  \subsection{Algorithms for Integer Operations}
  \subsection{Modular Exponentiation}
  
 \section{Primes and Greates Common Divisors}
  \subsection{Introduction}
  \subsection{Primes}
  \subsection{Trial Division}
  \subsection{The Sieve of Eratosthenes}
  \subsection{Conjectures and Open Problems About Primes}
  \subsection{Greatest Common Divisors and Least Common Multiples}
  \subsection{The Euclidean Algorithm}
  
 \section{Solving Congruences}
  \subsection{Introduction}
  \subsection{Linear Congruences}
  \subsection{The Chinese Remainder Theorem}
  \subsection{Computer Arithmetic with Large Integers}
  \subsection{Fermat's Little Theorem}
  \subsection{Pseudoprimes}
  \subsection{Primitive Roots and Discrete Logarithms}
  
 \section{Applications of Congruences}
  \subsection{Hashing Functions}
  \subsection{Pseudorandom Numbers}
  \subsection{Check Digits}
  
 \section{Cryptography}
  \subsection{Introduction}
  \subsection{Classical Cryptography}
  \subsection{PublicKey Cryptography}
  \subsection{The RSA Cryptosystem}
  \subsection{RSA Encryption}
  \subsection{RSA Dcryption}
  \subsection{RSA as a Public Key System}
  \subsection{Cryptographic Protocols}
  
 
\chapter{Induction and Recursion}
 \section{Mathematical Induction}
  \subsection{Introduction}
  \subsection{Mathematical Induction}
  \subsection{Why Mathematical Induction is Valid}
  \subsection{The Good and the Bad of Mathematical Induction}
  \subsection{Examples of Proofs by Mathematical Induction}
  \subsection{Mistaken Proofs By Mathematical Induction}
  \subsection{Guidelines for Proofs by Mathematical Induction}
  
 \section{Strong Induction and Well-Ordering}
  \subsection{Introduction}
  \subsection{Strong Induction}
  \subsection{Examples of Proofs Using Strong Induction}
  \subsection{Using Strong Induction in Computation Geometry}
  \subsection{Proofs Using the Well-Ordered Property}
  
 \section{Recursive Definitions and Structural Induction}
  \subsection{Introduction}
  \subsection{Recursively Defined Functions}
  \subsection{Recursively Defined Sets and Structures}
  \subsection{Structural Induction}
  \subsection{Generalized Induction}
  
 \section{Recursive Algorithms}
  \subsection{Induction}
  \subsection{Proving Recursive Algorithms Correct}
  \subsection{Recursion and Iteration}
  \subsection{The Merge Sort}
  
 \section{Program Correctness}
  \subsection{Introduction}
  \subsection{Program Verification}
  \subsection{Rules of Inference}
  \subsection{Conditional Statements}
  \subsection{Loop Invariants}
 
\chapter{Counting}
 \section{The Basics of Counting}
  \subsection{Introduction}
  \subsection{Basic Counting Principles}
  \subsection{More Complex Counting Problems}
  \subsection{The Subtraction Rule (Inclusion-Exclusion for Two Sets)}
  \subsection{The Division Rule}
  \subsection{Tree Diagrams}
  
 \section{The Pigeonhole Principle}
  \subsection{Introduction}
  \subsection{The Generalized Pigeonhole Principle}
  \subsection{Some Elegant Applications of the Pigeonhold Principle}
 
 \section{Permutations and Combinations}
  \subsection{Introduction}
  \subsection{Permutations}
  \subsection{Combinations}
  
 \section{Binomial Coefficients and Indentities}
  \subsection{The Binomial Theorem}
  \subsection{Pascal's Identify and Triangle}
  \subsection{Other Identities Involving Binomial Coefficients}
  
 \section{Generalized Permutations and Combinations}
  \subsection{Introduction}
  \subsection{Permutations with Repetition}
  \subsection{Combinations with Repetition}
  \subsection{Permutations with Indistinguishable Objects}
  \subsection{Distribuing Objects into Boxes}
  
 \section{Generating Permutations and Combinations}
  \subsection{Introduction}
  \subsection{Generating Permutations}
  \subsection{Generating Combinations}
 
\chapter{Discrete Probability}
 \section{An Introduction to Discrete Probability}
  \subsection{Introduction}
  \subsection{Finite Probability}
  \subsection{Probabilities of Complements and Unions of Events}
  \subsection{Probabilistic Reasoning}
  
 \section{Probability Theory}
  \subsection{Introduction}
  \subsection{Assigning Probabilities}
  \subsection{Probabilities of Complements and Unions of Events}
  \subsection{Conditional Probability}
  \subsection{Independence}
  \subsection{Bernoulli Trials and the Binomial Distribution}
  \subsection{Random Variables}
  \subsection{The Birthday Problem}
  \subsection{Monte Carlo Algorithms}
  \subsection{The Probabilistic Method}
  
 \section{Bayes' Theorem}
  \subsection{Introduction}
  \subsection{Bayes' Theorem}
  \subsection{Bayesian Spam Filters}
  
 \section{Expected Value and Variance}
  \subsection{Introduction}
  \subsection{Expected Values}
  \subsection{Linearity of Expectations}
  \subsection{Average-Case Computational Complexity}
  \subsection{The Geometric Distribution}
  \subsection{Independent Random Variables}
  \subsection{Variance}
 
\chapter{Advanced Counting Techniques}
 \section{Application of Recurrence Relations}
  \subsection{Introduction}
  \subsection{Modeling With Recurrence Relations}
  \subsection{Algorithms and Recurrence Relations}
  
 \section{Solving Linear Recurrence Relations}
  \subsection{Introduction}
  \subsection{Solving Linear Homogeneous Recurrence Relations with Constant Coefficients}
  \subsection{Linear Nonhomogeneous Recurrence Relations with Constant Coefficients}
  
 \section{Divide-and-Conquer Algorithms and Recurrence Relations}
  \subsection{Introduction}
  \subsection{Divide-and-Conquer Recurrence Relations}
  
 \section{Generating Functions}
  \subsection{Introduction}
  \subsection{Useful Facts About Power Series}
  \subsection{Counting Problems and Generating Functions}
  \subsection{Using Generating Functions to Solve Recurrence Relations}
  \subsection{Proving Indentities via Generating Functions}
  
 \section{Inclusion-Exclusion}
  \subsection{Introduction}
  \subsection{The Principle of Inclusion-Exclusion}
  
 \section{Applications of Inclusion-Exclusion}
  \subsection{Introduction}
  \subsection{An Alternative Form of Inclusion-Exclusion}
  \subsection{The Sieve of Eratosthenes}
  \subsection{The Number of Onto Functions}
  \subsection{Derangements}
 
\chapter{Relations}
 \section{Relations and Their Properties}
  \subsection{Introduction}
  \subsection{Functions as Relations}
  \subsection{Relations on a Set}
  \subsection{Combining Relations}
  
 \section{$n$-ary Relations and Their Applications}
  \subsection{Introduction}
  \subsection{$n$-ary Relations}
  \subsection{Databases and Relations}
  \subsection{Operations on $n$-ary Relations}
  \subsection{SQL}
  
 \section{Representing Relations}
  \subsection{Introduction}
  \subsection{Representing Relations Using Matrices}
  \subsection{Representing Relations Using Digraphs}
  
 \section{Closure of Relations}
  \subsection{Introduction}
  \subsection{Closures}
  \subsection{Paths in Directed Graphs}
  \subsection{Transitive Closures}
  \subsection{Warshall's Algorithm}
  
 \section{Equivalence Relations}
  \subsection{Introduction}
  \subsection{Equivalence Relations}
  \subsection{Equivalence Classes}
  \subsection{Equivalence Classes and Partitions}
  
 \section{Partial Orderings}
  \subsection{Introduction}
  \subsection{Lexicographic Order}
  \subsection{Hasse Diagrams}
  \subsection{Maximal and Minimal Elements}
  \subsection{Lattices}
  \subsection{Topological Sorting}
 
\chapter{Graphs}
 \section{Graphs and Graph Models}
  \subsection{Graph Models}
  
 \section{Graph Terminology and Special Types of Graphs}
  \subsection{Introduction}
  \subsection{Basic Terminology}
  \subsection{Bipartite Graphs}
  \subsection{Bipartite Graphs and Matchings}
  \subsection{Some Applications of Special Types of Graphs}
  \subsection{New Graphs from Old}
  
 \section{Representing Graphs and Graph Isomorphism}
  \subsection{Introduction}
  \subsection{Representing Graphs}
  \subsection{Adjacency Matrices}
  \subsection{Incidence Matrices}
  \subsection{Determing whether Two Simple Graphs are Isomorphic}
  
 \section{Connectivity}
  \subsection{Introduction}
  \subsection{Paths}
  \subsection{Connectedness in Unidirected Graphs}
  \subsection{How Connected is a Graph?}
  \subsection{Connectedness in Directed Graphs}
  \subsection{Paths and Isomorphism}
  \subsection{Counting Paths Between Vertices}
  
 \section{Euler and Hamilton Paths}
  \subsection{Introduction}
  \subsection{Euler Paths and Circuits}
  \subsection{Hamilton Paths and Circuits}
  \subsection{Applications of Hamilton Circuits}
  
 \section{Shortest-Path Problems}
  \subsection{Introduction}
  \subsection{A Shortest-Path Algorithm}
  \subsection{The Traveling Salesperson Problem}
  
 \section{Planar Graphs}
  \subsection{Introduction}
  \subsection{Kuratowski's Theorem}
  
 \section{Graph Coloring}
  \subsection{Introduction}
  \subsection{Applications of Graph Colorings}
 
\chapter{Trees}
 \section{Introduction to Trees}
  \subsection{Rooted Trees}
  \subsection{Trees as Models}
  \subsection{Properties of Trees}
  
 \section{Applications of Trees}
  \subsection{Introduction}
  \subsection{Binary Search Trees}
  \subsection{Decision Trees}
  \subsection{Game Trees}
  
 \section{Tree Traversal}
  \subsection{Introduction}
  \subsection{Universal Address Systems}
  \subsection{Traversal Algorithms}
  \subsection{Infix, Prefix, and Postfix Notation}
  
 \section{Spanning Trees}
  \subsection{Introduction}
  \subsection{Depth-First Search}
  \subsection{Breadth-First Search}
  \subsection{Backtracking Applications}
  \subsection{Depth-First Search in Directed Graphs}
  
 \section{Minimum Spanning Trees}
  \subsection{Introduction}
  \subsection{Algorithms for Minimum Spanning Trees}
 
\chapter{Boolean Algebra}
 \section{Boolean Functions}
  \subsection{Introduction}
  \subsection{Boolean Expressions and Boolean Functions}
  \subsection{Identities of Boolean Algebra}
  \subsection{Duality}
  \subsection{The Abstract Definition of a Boolean Algebra}
  
 \section{Representing Boolean Functions}
  \subsection{Sum-of-Products Expansions}
  \subsection{Functional Completeness}
  
 \section{Logic Gates}
  \subsection{Introduction}
  \subsection{Combinations of Gates}
  \subsection{Examples of Circuits}
  \subsection{Adders}
  
 \section{Minimization of Circuits}
  \subsection{Introduction}
  \subsection{Karnaugh Maps}
  \subsection{Don't Care Conditions}
  \subsection{The Quine-McCluskey Method}

\chapter{Modeling Computation}
 \section{Languages and Grammars}
  \subsection{Introduction}
  \subsection{Phrase-Structure Grammars}
  \subsection{Types of Phrase-Structure Grammars}
  \subsection{Derivation Trees}
  \subsection{Backus-Naur Form}
  
 \section{Finite-State Machines with Output}
  \subsection{Introduction}
  \subsection{Finite-State Machines with Outputs}
  

 \section{Finite-State Machines with No Output}
  \subsection{Introduction}
  \subsection{Set of Strings}
  \subsection{Finite-State Automata}
  \subsection{Language Recognition by Finite-State Machines}
  \subsection{Nondeterministic Finite-State Automata}
  
 \section{Language Recognition} 
  \subsection{Introduction}
  \subsection{Kleene's Theorem}
  \subsection{Regular Sets and Regular Grammars}
  \subsection{More Powerful Types of Machines}
  
 \section{Turing Machines}
  \subsection{Introduction}
  \subsection{Definition of Turing Machines}
  \subsection{Using Turing Machines to Recognize Sets}
  \subsection{Computing Functions with Turing Machines}
  \subsection{Different Types of Turing Machines}
  \subsection{The Church-Turing Thesis}
  \subsection{Computational Complexity, Computability, and Decidability}

\end{document}